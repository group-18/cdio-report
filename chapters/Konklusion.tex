\chapter{Konklusion}
Som afslutning på det her projekt kan vi konkludere, at formålet med dette CDIO projekt er opnået. 
I dette projekt har vi analyseret, designet, implementeret og testet et program
At analysere, designe, implementere, og teste et program, der simulere et terningsspil mellem to spillere.
Kravene til programmet, herunder størstedelen af de ekstra funktioner stillet i opgaveformuleringen.

Vi gruppens medlemmer, har gennem analysen, udviklingen, samt test af projektet forsøgt at arbejde på en agil måde og dermed følge en arbejdsprocess, der tager udgangspunkt i UP (Unified Process). \\

Resultatet af vores arbejde er tilfredsstillende.
Selvom vi fik opnået et godt stykke arbejde, var der visse områder som vi havde i sinde at forbedre, hvis tidsperioden tilladte dette.
I projektformuleringen var der tale om en såkaldt GUI (Graphical User Interface), en brugergrænseflade, som var vedhæftet med i opgaveformuleringen.
GUI’en var en del af et matadorspil, hvilket vi kunne anse som en inspirationskilde.
Desværre mødte vi lidt besvær under forsøget, nemlig at få pillet noget brugbart ud fra kildekoden.
Koden til GUI’en var nemlig en anelse indviklet, og da vi ikke har nogen form for tidligere erfaring med JAVA, havde vi hellere ikke mulighed for at konstruerer en GUI fra bunden.
Resultatet af dette var, at vi i gruppen fravalgte, at gøre brug af nogen form for GUI.
Selvom vi kunne forestille os, at ved et større projekt, ville vi godt kunne prioritere GUI en anelse mere.
Et andet område med plads til forbedring, er planlægningen generelt. 
På trods af at dette ikke er første gang, hvor vi har arbejdet i grupper, så føler vi at en bedre planlægningen af projektet, ville have gjort en forskel hvad angår tidsfordeling, og kvaliteten af opgaverne. \\

Alt i alt, er vi tilfredse med resultatet af dette CDIO 1 projekt.
Vi mener, at vi har fået et godt førstehåndsindtryk samt en klar ide af, hvordan og hvorledes de kommende CDIO projekter skal gribes an.