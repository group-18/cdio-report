\chapter{Konklusion}
Som afslutning på det her projekt kan vi konkludere at formål med dette CDIO projekt er opnået. At analysere, designe, implementere, og teste et programmeringsprojekt der simulere et terningsspil mellem to spillere har fundet sted i dette projekt, og med fremragende resultater. Mere specifikt, er kravene stillet til programmet i form af kundens vision opfyldt, herunder største delen af de ekstra funktioner stillet i opgaveformuleringen. Vi, gruppens medlemmer, udviklere i spilfirmaet IOOuterActive, har gennem analysen, udviklingen samt test af projektet forsøgt at arbejde på en agil måde og dermed følge en arbejdsprocess der tager udgangspunkt i UP (Unified Process). \\

Resultatet af vores arbejde er tilfredsstillende, og selvom at vi fik opnået et godt stykke arbejde, så var der visse områder som vi havde i sindet, at forbedre hvis tidsperioden tilladte. I projektformuleringen var der tale om en såkaldt GUI (Graphical User Interface), på dansk, en brugergrænseflade, som var vedhæftet med i opgaveformuleringen. GUI’en var en del af et matadorspil, hvilket vi kunne anse som en inspirationskilde. Desværre mødte vi lidt besvær under forsøget, nemlig at pille et eller andet brugbart fra kildekoden, dog uden held. Koden til GUI’en var nemlig en anelse indviklet og da vi ikke har nogen form for tidligere erfaring med JAVA, havde vi hellere ikke mulighed for at konstruerer en GUI fra bunden. Resultatet af dette var, at vi i gruppen fravalgte at gøre brug af nogen form for GUI, grundet det korte tidsrum. Et andet område med plads til forbedring er planlægningen generelt. På trods af at dette ikke er første gang hvor vi har arbejdet i grupper, så føler vi at en bedre planlægningen af projektet vil have gjort en forskel hvad angår tidsfordeling af opgaverne. \\

Alt i alt er vi tilfredsse med resultatet af dette CDIO 1 projekt, og vi mener at vi har fået et godt førstehåndsindtryk samt en klar ide af hvordan og hvorledes de kommende CDIO projekter skal gribes an.