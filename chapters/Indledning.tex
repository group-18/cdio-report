\chapter{Indledning}
I dette rapport vil vi gennemgå en opgave, som spilfirmaet \textit{IOOuterActive} har modtaget, og vi, som udviklere for dette firma, vil skabe dette spil, som opgaven beskriver.
Opgaven vil i denne rapport optræde som 'CDIO1'.
\\\\CDIO1 er et projekt, der går ud på at skrive et terningespil.
Vi har i denne opgave forsøgt at arbejde på en agil måde og har derved valgt at arbejde med UP (Unified Process), som i sin helhed går ud på at dele projektet op i iterationer (normalt vil disse vare 2-6 uger, men i dette korte projekt har vi ikke tidsbestemte iterationer).
I projektet har vi primært brugt brugt følgende software:
\begin{enumerate}
    \item \textbf{Visual Studio Code}
    \\ Vi valgte fra starten at skrive i LaTeX, for at lette skrivningsprocessen og få en homogen rapport.
    Samtidig er filtypen .tex understøttet af .git.
    \item \textbf{Tower}
    \\Tower er et af de mange GIT-UI på markedet, og det fungerer eminent til de funktioner, vi har benyttet os af.
    \item \textbf{IntelliJ}
    \\Det endelig program er skrevet i IntelliJ sideløbende med Eclipse for at teste, om slutproduktet fungerer for flere forskellige maskiner og i forskellige programmer.
    \item \textbf{Asana}
    \\Asana er et projektstyringsprogram, hvor man kan skabe opgaver og tildele personer til opgaver med eventuelle deadlines.
    \item \textbf{Slack}
    \\Slack er en \textit{digital workspace}, hvor man kan kommunikere med projekts medarbejdere.
\end{enumerate}