\chapter{Implementering}
    I vores arbejde med at kode dette program, har vi startet ud med at 
    se bort fra de ekstra krav vi har fået, og blot fået implementeret kerneprogrammet.
    \\ Kerneprogrammet er programmet, der opretter to spillere, 
    og to terninger, og den spiller der først når 40 point vinder. 
    Vi besluttede os for at lave et while loop, således at programmet fortsætter med at køre,
    indtil vi har opnået en bestemt score.
    \\ Dermed er kerne-delen af programmet blevet lavet. 
    Herefter begyndte vi på, at arbejde med, at implementere de ekstra krav, der
    er stillet. 
    Dette blev grundigt overvejet, før valgtet af form for kode, blev bestemt.
    \\ Vi fandt ud af, at alle ekstra krav ville blive løst, ved at benytte if-else sætninger,
    hvor i krav kriterierne indgår.
    De første 3 krav var relativt enkle at opnå, da disse bare bestod af simple
    if-else sætninger.
    Det fjerde krav, var sværere at få implementeret. Kravet lyder på, at man kun kan vinde på et par.
    Vi har valgt at kode programmet sådan, så hvis man får over 40 point, 
    men ikke slår en double, vil programmet sætte din score til 40.
    Dette har vi gjort, da det er ligegyldigt om man har 40 eller 100 point,
    ligeså snart man har 40 point skal man bruge et par for at vinde.
    Ved at stille det op på denne måde, sørger vi for, at while loopet fortsætter,
    til at der er en spiller der både har en score>=40, og at han har slået et par
    i denne runde. \\

    \section{Brugervejledning for programmet}
    Programmet er meget simpelt sat op, og har en masse System.out.println
    der fortæller brugeren præcis hvad der foregår, og hvad han skal gøre for
    at programmet fortsætter. 
    \\ Programmet starter med at printe alle spillets regler,
    således at brugeren kan se reglerne. Derefter beder programmet brugeren om at
    indtaste navn på spiller 1, og navn på spiller 2. Herefter vil programmet
    blive ved med at spørge brugeren, om at skrive "roll", indtil brugeren taster dette.
    Programmet vil herefter printe et terningeslag. Summen af slaget,
    og stillingen mellem de to spillere, i tilfælde af at der er slået et par,
    vil programmet skrive at det er den samme spillers tur.
    Programmet bliver ved med dette, indtil en spiller opnår over 40 point,
    efter dette vil programmet skrive, at man skal slå et par for at vinde.
    Dette bliver den ved med indtil at en af spillerne slår et par, der ikke
    er par 1, da par 1 selvfølgelig nulstiller spillerens score.